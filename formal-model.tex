\section{A Formal Model of Typed Clojure}

We describe our type system in a similar way to Typed Racket (\citet{TF10}),
with differences highlighted in $\trdiff{\text{blue}}$.
The main judgement 

$
{\judgementsevcol {\mmenv}
                 {\taenv}
                 {\propenv}
               {\hastype {\e{}} {\t{}}}
  {\filterset {\thenprop {\prop{}}}
              {\elseprop {\prop{}}}}
  {\object{}}
  {\mmenvp{}}}
$

says expression \e{} is of type \t{} in the 
multimethod environment $\trdiff{\mmenv{}}$, the \emph{tools.analyzer} environment
$\trdiff{\taenv{}}$
and proposition environment $\propenv{}$, with 
then proposition {\thenprop {\prop{}}}, else proposition {\elseprop {\prop{}}},
object \object{} and new multimethod environment
$\trdiff{\mmenvp{}}$.


\subsection{Higher-rank polymorphism}

Typed Clojure supports f-bounded polymorphism and higher-rank
types.
% why include them?
% examples, involving generic type operators for monads & conduits
% useful for generic programming, similar to Hashell type classes

\subsection{Using negative filters}

Occurrence typing plays an important role in Typed Racket and Typed Clojure.
By maintaining a \emph{proposition environment} of propositions relating types to
bindings, we can update bindings with more accurate types as programs progress.
It follows that there is some correspondence between propositions and types,
characterised by the \emph{update} function, which takes a type and a proposition
and returns a type which updates the input type using the proposition.

There is a straightforward relationship between ``positive'' propositions and types.
For example 
{\tt (update Number (is Integer 0))}
updates Number by Integer, which is Integer, because Integer <: Number.

The relationship between ``negative'' propositions and types is not always obvious.
A common proposition in Typed Clojure is (! (U nil false) a): the proposition that
local binding ``a'' is \emph{not} of type (U nil false).
This problem is most visible in expressions like {\tt (filter identity coll)}, where
``identity'' has a ``then'' proposition that has negative information: (! (U nil false) 0),
which reads, the 0th argument of identity does not contain (U nil false).

\subsection{Multimethods}

Clojure provides multimethods which can dispatch on the result of an
arbitrary function, the \emph{dispatch function}. \emph{Methods} are
added to the multimethod
which are associated with a \emph{dispatch value}, which is compared to the
result of apply the dispatch function to the argument passed to the
multimethod. If this comparison unambiguously dispatches to one method,
that method is then called with the current arguments.

Occurrence typing, in particular when extended with paths,
for checking method definitions for arbitrary dispatch functions.

\begin{lstlisting}
(ns example.mm
  (:require 
    [clojure.core.typed :refer [ann]]))

(ann my-mm [Any -> Number])
(defmulti my-mm class)

(defmethod my-mm Float [f] (inc f))
\end{lstlisting}

\begin{lstlisting}
(ann clojure.core/class
  [Any -> (U nil Class)
   :object {:id 0, :path [Class]}])
\end{lstlisting}

\begin{verbatim}
TODO
- example of normal MM usage
- explain descrepency btwn TC syntax and formal
  TR calculs syntax. eg. :object/:filter 
    vs over arrow
\end{verbatim}

\subsection{Extending occurrence typing}

\subsection{Java interoperability}

As Clojure supports Java interoperability, Typed Clojure has some
integration with Java's type system. Calls to Java methods from
Clojure can be type checked by Typed Clojure.

We do not support Java Generics for several reasons. Firstly
Clojure programmers already use \emph{type hints} to resolve
Java methods which does not interact with generics. Secondly
it would complicate the type system to need to consider advanced
Generics features like existentials and f-bounded polymorphism
which has complicated other type systems like Scala's.
Finally Java interop is rare in Clojure code, so the extra
work needed would not be worth the effort compared to the number
of new bugs core.typed catches.

Thanks to type erasure, only unparameterised classes, primitive
types and array classes live at runtime. This simplifies 
inference for \emph{tools.analyzer}, which uses type hints
to resolve ambiguities in Java interop like overloaded methods.

\subsection{Clojure type hints}

Clojure comes with a small type system that helps improve
performance in Clojure code: type hints. They are used to help
the compiler resolve Java methods and prevent unboxing in
certain situations. Tags are propagated bidirectionally
and the programmer is free to add them where they like.

\subsubsection{Arrays}
\label{sec:arrays}

Supporting statically sound interactions with Java arrays is a goal
of Typed Clojure. This is complicated by Java's decision to make
arrays covariant in their argument, a well documented source of static
unsoundness. Bracha~\cite{Bra98} summarises Java's approach to maintaining
soundness at runtime, which involves all array writes being checked by
runtime assertions.

This approach fits Java's type system, but we can do better in a more powerful
type system like Typed Clojure. Our goal is to catch all type-incorrect array
writes at compile time so the type system can do more to help Clojure programmers
use arrays, especially those being passed from foreign Java code.

Our basic approach is to make our array types \emph{bivariant}. Array types
look like {\ArrayTwo {\t{w}} {\t{r}}} and
are reminiscent of functions or pipes: having a contravariant parameter for input (writing)
and a covariant parameter for output (reading).
This type can write type {\t{w}} and read type {\t{r}}.

Most commonly, an array type is invariant in its parameter; it can
write and read input of the same type.
We can get the same effect by setting our input and output
parameters to the same type. For example, {\ArrayTwo {\Number} {\Number}}
(or equivalently, {\Array {\Number}})
in Typed Clojure is similar to invariant array types of \Number in languages like Scala.

The biggest gain in using a separate input parameter is the ability
to specify \emph{read-only} arrays. Crucially, our type system features an
explicit bottom type \lstinline|Nothing|, enabling a read-only \lstinline|Number| array
to be of type \lstinline|(Array2 Nothing Number)|.

To realise why defining read-only arrays are useful, we need to examine
what makes array covariance unsound in Java.
\begin{verbatim}
FIXME
Array covariance about the type of an array so the consumer
of an array cannot tell the actual type of the array when examining a type
signature.
\end{verbatim}

\begin{lstlisting}
...
public static Number[] getNumberArray() {
  Number[] n = new Integer[10];
  return n;
}
...
\end{lstlisting}

To the casual consumer \emph{getNumberArray} returns an array that can both
read and write \lstinline|Number|s. However it is clear from the implementation
that attempting to write say a \lstinline|Double| to this array will result
in a runtime error.

\begin{verbatim}
...
Number[] myArray = getNumberArray();
myArray[0] = 1.1;
/* Exception in thread "main" 
   java.lang.ArrayStoreException: 
   java.lang.Double */
...
\end{verbatim}

Notice that this is a runtime error, and Java's type system has not helped
statically prevent it.
This could cause a similar issue for other statically-typed languages offering
interoperability with Java. 

To prevent these sorts of runtime exceptions in Typed Clojure, we declare
all arrays from unknown sources to be \emph{read-only}. Put differently,
the only way to define a writeable array is to create it in type-checked Clojure
code.

\begin{lstlisting}
(let [n (CovariantArray/getNumberArray)]
  (aset n 0 1.1))

; Polymorphic static method clojure.lang.RT/aset could not be 
; applied to arguments:
; Domains: 
;         (Array2 i o) clojure.core.typed/AnyInteger i
; 
; Arguments:
;         (Array2 Nothing java.lang.Number) int (Value 1.1)
; 
; with expected type:
;         Any
\end{lstlisting}

The type inferred for the local \lstinline|n| is \lstinline|(Array2 Nothing Number)|
which tells the type system: it is never safe to write to this array, but
it is safe to assume \lstinline|Number|s can be read from this array.

To emphasise, Typed Clojure throws a static type error. Errors like this help Clojure programmers
use foreign Java libraries more correctly.

\begin{verbatim}
Note that Java libraries are often large 
and complex and programmers will probably
enjoy the extra help from the type system.
\end{verbatim}

\subsection{Handling null}
\label{sec:null}

Probably the most common pain point in Java programming is dealing with \emph{null}.
It is crucial that Typed Clojure deals with \emph{null} intelligently so that
all potentially erroneous interactions with \emph{null} in typed code are caught at compile time.
We also want to represent \emph{null} in a way that is natural to Clojure programmers
and approximates their mental model when reasoning about Clojure code.

Firstly, we separate the concepts of \emph{null} and reference types at the type level.
This is unlike Java, where a reference type, like \emph{java.lang.Number}, includes \emph{null}.
We provide a singleton type \Nil{}, which contains just the value \lstinline|nil|,
Clojure's equivalent of \emph{null}. 

Armed with general unions and the type \lstinline|nil|, 
we make Typed Clojure's reference types \emph{non-nullable}.
A Java type like \emph{java.lang.Number} is written \lstinline|(U nil java.lang.Number)|
in Typed Clojure.

Typed Racket already provides what is needed to write such types; indeed this is a repurposing
of a common Typed Racket idiom of approximating an Option or Maybe type by
making a union that includes \emph{\#f}, Racket's only false value.
However, like handling covariant arrays, the subtlety lies in our interactions with foreign
Java code.

A set of conversion rules define how to convert a Java type signature into a null-safe
Clojure version. These rules are designed to be conservative and can be explicitly overridden
by programmers. Method parameters are non-nullable, while return types are nullable.
Field types are nullable. Constructor parameters are non-nullable, and return types
are also non-nullable: it is a Java invariant that all constructors return an instance
of the class they are constructing.

\begin{verbatim}
TODO 
- write conversion rules for 
  Java type -> Clojure type
\end{verbatim}
