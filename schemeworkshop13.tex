%-----------------------------------------------------------------------------
%
%               Template for sigplanconf LaTeX Class
%
% Name:         sigplanconf-template.tex
%
% Purpose:      A template for sigplanconf.cls, which is a LaTeX 2e class
%               file for SIGPLAN conference proceedings.
%
% Guide:        Refer to "Author's Guide to the ACM SIGPLAN Class,"
%               sigplanconf-guide.pdf
%
% Author:       Paul C. Anagnostopoulos
%               Windfall Software
%               978 371-2316
%               paul@windfall.com
%
% Created:      15 February 2005
%
%-----------------------------------------------------------------------------


\documentclass{sigplanconf}

% The following \documentclass options may be useful:

% preprint      Remove this option only once the paper is in final form.
% 10pt          To set in 10-point type instead of 9-point.
% 11pt          To set in 11-point type instead of 9-point.
% authoryear    To obtain author/year citation style instead of numeric.

\usepackage{mmm}
\usepackage{mathpartir}
\usepackage{clj-grammar}
%\usepackage{url}
%\usepackage[style=alphabetic, sorting=nyt]{biblatex}
%\addbibresource{bibliography.bib}

\usepackage{listings}
\lstset{ %
  language=Lisp,                % choose the language of the code
  columns=fixed,basewidth=.5em,
  basicstyle=\small\ttfamily,       % the size of the fonts that are used for the code
  %numbers=left,                   % where to put the line-numbers
  %numberstyle=\small\ttfamily,      % the size of the fonts that are used for the line-numbers
  %stepnumber=1,                   % the step between two line-numbers. If it is 1 each line will be numbered
  %numbersep=5pt,                  % how far the line-numbers are from the code
  %backgroundcolor=\color{white},  % choose the background color. You must add \usepackage{color}
  %showspaces=false,               % show spaces adding particular underscores
  showstringspaces=false,         % underline spaces within strings
  %showtabs=false,                 % show tabs within strings adding particular underscores
  frame=single,           % adds a frame around the code
  %tabsize=2,          % sets default tabsize to 2 spaces
  captionpos=t,           % sets the caption-position to bottom
  breaklines=true,        % sets automatic line breaking
  breakatwhitespace=true,    % sets if automatic breaks should only happen at whitespace
  %escapeinside={\%*}{*)},          % if you want to add a comment within your code
}


\usepackage{amsmath}


\begin{document}

\special{papersize=8.5in,11in}
\setlength{\pdfpageheight}{\paperheight}
\setlength{\pdfpagewidth}{\paperwidth}

\conferenceinfo{CONF 'yy}{Month d--d, 20yy, City, ST, Country} 
\copyrightyear{20yy} 
\copyrightdata{978-1-nnnn-nnnn-n/yy/mm} 
\doi{nnnnnnn.nnnnnnn}

% Uncomment one of the following two, if you are not going for the 
% traditional copyright transfer agreement.

%\exclusivelicense                % ACM gets exclusive license to publish, 
                                  % you retain copyright

%\permissiontopublish             % ACM gets nonexclusive license to publish
                                  % (paid open-access papers, 
                                  % short abstracts)

\titlebanner{banner above paper title}        % These are ignored unless
\preprintfooter{short description of paper}   % 'preprint' option specified.

\title{Title Text}
\subtitle{Subtitle Text, if any}

\authorinfo{Name1}
           {Affiliation1}
           {Email1}
\authorinfo{Name2\and Name3}
           {Affiliation2/3}
           {Email2/3}

\maketitle

\begin{abstract}
This is the text of the abstract.
\end{abstract}

\category{CR-number}{subcategory}{third-level}

% general terms are not compulsory anymore, 
% you may leave them out
%\terms
%term1, term2

\keywords
keyword1, keyword2

\section{Introduction}

Typed Clojure is a gradual type system for Clojure, based on the
static portion of Typed Racket.
This paper gives an overview of Typed Clojure, concentrating on extensions
that could interest Typed Racket users.

% what are the goals of Typed Clojure?
% where do they differ from Typed Racket? why?

% Things todo
% - small calculus based on previous TR formalisation
%   that demos the ``flow'' environment & not/intersection type
% - discussion and examples for flow filters
% - discussion on the not/difference type, real code where it helps
% - higher-rank polymorphism, show what it enables (eg. monads)
%   - also show more complicated example, making an extensible
%     type for clojure.core/conj, and why it's not straightforward
% - discussion on limitations of local type inference
% - demonstration of how we check multimethods
% - dealing with array covariance
% - note on how we avoid null-pointer exceptions
% - discussion on why Typed Clojure only uses the ``static'' part
%   of Typed Racket
% - the equality filter problem (and draft solution?), a proposition
%   that knows about binding aliasing. Sounds straightforward?


\section{Using negative filters}

Occurrence typing plays an important role in Typed Racket and Typed Clojure.
By maintaining a \emph{proposition environment} of propositions relating types to
bindings, we can update bindings with more accurate types as programs progress.
It follows that there is some correspondence between propositions and types,
characterised by the \emph{update} function, which takes a type and a proposition
and returns a type which updates the input type using the proposition.

There is a straightforward relationship between ``positive'' propositions and types.
For example 
{\tt (update Number (is Integer 0))}
updates Number by Integer, which is Integer, because Integer <: Number.

The relationship between ``negative'' propositions and types is not always obvious.
A common proposition in Typed Clojure is (! (U nil false) a): the proposition that
local binding ``a'' is \emph{not} of type (U nil false).
This problem is most visible in expressions like {\tt (filter identity coll)}, where
``identity'' has a ``then'' proposition that has negative information: (! (U nil false) 0),
which reads, the 0th argument of identity does not contain (U nil false).

\section{Extending occurrence typing}

\begin{figure*}
$$
\begin{altgrammar}
  \expd{}, \expe{} &::=& (\expe{}\ \expe{}) \alt \abs{\x a}{\expe{}}
                &\mbox{Expressions}
%  \c{} &::=& \assoc{} \alt \dissoc{} \alt \get{}
%              &\mbox{Constants}
\end{altgrammar}
$$
\end{figure*}

%$$
%\begin{tdisplay}{Evaluation Contexts}
%  \begin{altgrammar}
%    \E{} &::=& [ ] % application rules
%              \alt (\c{}\ \overrightarrow{\v{}}\ \E{}\ \overrightarrow{\exp{}}) % eval arguments left-to-right
%              % map rules
%              \alt \{\overrightarrow{\v{}\ \v{}}\ \E{}\ \exp{}\ \overrightarrow{\exp{}\ \exp{}} \} % key first
%              \alt \{\overrightarrow{\v{}\ \v{}}\ \v{}\ \E{}\ \overrightarrow{\exp{}\ \exp{}} \}   % value next
%              &\mbox{Evaluation Contexts}
%  \end{altgrammar}
%\end{tdisplay}
%$$ 


%\appendix
%\section{Appendix Title}
%
%This is the text of the appendix, if you need one.

%\acks
%
%Acknowledgments, if needed.

% We recommend abbrvnat bibliography style.

%\bibliographystyle{abbrvnat}
%
%% The bibliography should be embedded for final submission.
%
%\begin{thebibliography}{}
%\softraggedright
%
%\bibitem[Smith et~al.(2009)Smith, Jones]{smith02}
%P. Q. Smith, and X. Y. Jones. ...reference text...
%
%\end{thebibliography}


\end{document}

%                       Revision History
%                       -------- -------
%  Date         Person  Ver.    Change
%  ----         ------  ----    ------

%  2013.06.29   TU      0.1--4  comments on permission/copyright notices

